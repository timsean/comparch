\documentclass[10pt]{article}

\usepackage{graphicx}
\usepackage{float}
\usepackage[T1]{fontenc}
\usepackage[titletoc]{appendix}
\usepackage[margin=0.7in]{geometry}
\usepackage{blindtext}
\usepackage[english]{babel}
\usepackage{multirow}
\usepackage{courier}
\usepackage{fancyhdr}
\usepackage{makecell}
\usepackage{caption}
\usepackage{rotating}
\usepackage{pdflscape}
\pagestyle{fancy}
\lhead{Tim Yao - ty252}

\renewcommand{\thesubsection}{\thesection.\alph{subsection}}

\usepackage{listings}
\usepackage{color}

\definecolor{dkgreen}{rgb}{0,0.6,0}
\definecolor{gray}{rgb}{0.5,0.5,0.5}
\definecolor{mauve}{rgb}{0.58,0,0.82}

\lstset{frame=tb,
  language=Verilog,
  aboveskip=3mm,
  belowskip=3mm,
  showstringspaces=false,
  columns=flexible,
  keepspaces=true,
  basicstyle={\small\ttfamily},
  keywordstyle=\color{blue},
  commentstyle=\color{dkgreen},
  stringstyle=\color{mauve},
  breaklines=false,
  breakatwhitespace=true,
  tabsize=2
}

% For monospace stuff


\title{ECE 4750 PSET 2}
\author{Tim Yao (ty252)}
\date{Oct 10, 2015}

\begin{document}
\maketitle
\newcommand*{\tableindent}{\hspace*{0.3cm}}%

\section{PARCv1 Instruction Cache} 
\subsection{Categorizing Cache Misses}
\begin{figure}[H]
\centering
\begin{tabular}{clcc}
\hline
\textbf{Addr} &\textbf{Instruction}& \textbf{Iteration 1} & \textbf{Iteration 2}\\
\hline
& loop: & & \\
\hline
0x108 & \tableindent addiu r1, r1, -1 & compulsory & \\
\hline
0x10c & \tableindent addiu r2, r2, -1 & 		   & \\
\hline
0x110 & \tableindent j foo			  & compulsory & conflict \\
\hline
	  & ...							  & 		   & \\
\hline
	  & foo:						  & 		   & \\
\hline
0x218 & \tableindent addiu r6, r6, 1  & compulsory & conflict \\
\hline
0x21c & \tableindent bne r1, r0, loop & 	       & \\
\hline
\end{tabular}
\caption{Cache Miss Type}
\end{figure}

\subsection{Average Memory Access Latency}
Looking at iteration 2, we can see that there are 2 misses out of the 5 instructions. Therefore the miss rate for 64 iterations of the loop is 0.4.\\
The average memory access latency is: \\
AMAL = (Hit Time) + (Miss Rate $\times$ Miss Penalty)\\
AMAL = 1 + (0.4 $\times$ 5)\\
AMAL = 3 cycles\\
The AMAL is dominated by conflict misses, as shown by the miss chart above. Compulsory misses only occur on the first iteration of the loop.

\subsection{Set-Associativity}
The cache performance will increase significantly, because there will no longer be conflict misses during the loop. With this new cache microarchitecture, only compulsory misses will be left. 

\cleardoublepage
\section{Page-Based Memory Translation}

\subsection{Two-Level Page Tables}
The 16-bit virtual address is used as the following:
\begin{figure}[H]
\centering
\begin{tabular}{|c|c|c|}
\hline
Bits 14-15 & Bits 12-13 & Bits 0-11 \\
\hline
XX & XX & XXXXXXXXXXXX \\
\hline
\multicolumn{2}{|c|}{Virtual Page Number} & Page Offset \\
\hline
L1 Index & L2 Index & Page Offset \\
\hline
\end{tabular}
\caption{Virtual Address Usage}
\end{figure}
Page Tables:
\begin{figure}[H]
\centering
{
\begin{tabular}{@{\extracolsep{3pt}}lcc@{}}
\Xhline{2\arrayrulewidth}
& \multicolumn{2}{c}{\textbf{Page-Table Entry}} \\
\cline{2-3}
\textbf{Paddr of PTE} & \textbf{Valid} & \textbf{Paddr}\\
\Xhline{2\arrayrulewidth}
0xffffc & 1 & 0xfffe0 \\
\hline
0xffff8 & &  \\
\hline
0xffff4 & &  \\
\hline
0xffff0 & 1 & 0xfffb0 \\
\Xhline{2\arrayrulewidth}
0xfffec & 1 & 0x05000 \\
\hline
0xfffe8 & 1 & 0x07000 \\
\hline
0xfffe4 & &  \\
\hline
0xfffe0 & & \\
\Xhline{2\arrayrulewidth}
0xfffdc & & \\
\hline
0xfffd8 & &  \\
\hline
0xfffd4 & &  \\
\hline
0xfffd0 & & \\
\Xhline{2\arrayrulewidth}
0xfffcc & & \\
\hline
0xfffc8 & &  \\
\hline
0xfffc4 & &  \\
\hline
0xfffc0 & & \\
\Xhline{2\arrayrulewidth}
0xfffbc & 1 & 0x01000\\
\hline
0xfffb8 & 1 & 0x04000\\
\hline
0xfffb4 & 1 & 0x00000 \\
\hline
0xfffb0 & & \\
\Xhline{2\arrayrulewidth}
\end{tabular}
}
\caption{Contents of Physical Memory with Page Tables}
\end{figure}

\subsection{Translation-Lookaside Buffer}
\begin{figure}[H]
\centering
{
\begin{tabular}{@{\extracolsep{3pt}}ccccccccc@{}}
\hline
& & & & \textbf{Total} & & & & \\
\textbf{Transaction} & & \textbf{Page} & & \textbf{Num Mem} & \multicolumn{2}{c}{\textbf{TLB Way 0}} & \multicolumn{2}{c}{\textbf{TLB Way 1}} \\
\cline{6-7}
\cline{8-9}
\textbf{Address} & \textbf{VPN} & \textbf{Offset} & \textbf{m/h} & \textbf{Accesses} & \textbf{VPN} & \textbf{PPN} & \textbf{VPN} & \textbf{PPN} \\
\hline
0xeff4 & 0xe & 0xff4 & m & 3 & - & - & - & - \\
\hline
0x2ff0 & 0x2 & 0xff0 & m & 3 & 0xe & 0x07 & & \\
\hline
0xeff8 & 0xe & 0xff8 & h & 1 &  &  & 0x2 & 0x04 \\
\hline
0x2ff4 & 0x2 & 0xff4 & h & 1 & & & & \\
\hline
0xeffc & 0xe & 0xffc & h & 1 & & & & \\
\hline
0x2ff8 & 0x2 & 0xff8 & h & 1 & & & & \\
\hline
0xf000 & 0xf & 0x000 & m & 3 & & & & \\
\hline
0x2ffc & 0x2 & 0xffc & h & 1 & 0xf & 0x05 & & \\
\hline
0xf004 & 0xf & 0x004 & h & 1 & & & & \\
\hline
0x3000 & 0x3 & 0x000 & m & 3 & & & & \\
\hline
0xf008 & 0xf & 0x008 & h & 1 & & & 0x3 & 0x01 \\
\hline
0x3004 & 0x3 & 0x004 & h & 1 & & & 0x2 & 0x04 \\
\hline
0xf00c & 0xf & 0x00c & h & 1 & & & & \\
\hline
0x3008 & 0x3 & 0x008 & h & 1 & & & & \\
\hline
\end{tabular}
}
\caption{TLB Contents Over Time}
\end{figure}

\cleardoublepage
\section{Impact of Cache Access Time and Replacement Policy}

\subsection{Miss Rate Analysis}
\begin{figure}[H]
\centering
\begin{tabular}{@{\extracolsep{3pt}}cccccccccccc@{}}
\Xhline{2\arrayrulewidth}
\textbf{Transaction} & & & & & & & & & & & \\
\textbf{Address} & \textbf{tag} & \textbf{idx} & \textbf{m/h} & \textbf{L0} & \textbf{L1} & \textbf{L2} & \textbf{L3} & \textbf{L4} & \textbf{L5} & \textbf{L6} & \textbf{L7}\\
\Xhline{2\arrayrulewidth}
0x024 & 0x0 & 0x2 & m &  -  &  -  &  -  &  -  &  -  &  -  &  -  &  -  \\
0x030 & 0x0 & 0x3 & m &     &     & 0x0 &     &     &     &     &     \\
0x07c & 0x0 & 0x7 & m &     &     &     & 0x0 &     &     &     &     \\
0x070 & 0x0 & 0x7 & h &     &     &     &     &     &     &     & 0x0 \\
0x100 & 0x2 & 0x0 & m &     &     &     &     &     &     &     &     \\
0x110 & 0x2 & 0x1 & m & 0x2 &     &     &     &     &     &     &     \\
0x204 & 0x4 & 0x0 & m &     & 0x2 &     &     &     &     &     &     \\
0x214 & 0x4 & 0x1 & m & 0x4 &     &     &     &     &     &     &     \\
0x308 & 0x6 & 0x0 & m &     & 0x4 &     &     &     &     &     &     \\
0x110 & 0x2 & 0x1 & m & 0x6 &     &     &     &     &     &     &     \\
0x114 & 0x2 & 0x1 & h &     & 0x2 &     &     &     &     &     &     \\
0x118 & 0x2 & 0x1 & h &     &     &     &     &     &     &     &     \\
0x11c & 0x2 & 0x1 & h &     &     &     &     &     &     &     &     \\
0x410 & 0x8 & 0x1 & m &     &     &     &     &     &     &     &     \\
0x110 & 0x2 & 0x1 & m &     & 0x8 &     &     &     &     &     &     \\
0x510 & 0xa & 0x1 & m &     & 0x2 &     &     &     &     &     &     \\
0x110 & 0x2 & 0x1 & m &     & 0xa &     &     &     &     &     &     \\
0x610 & 0xc & 0x1 & m &     & 0x2 &     &     &     &     &     &     \\
0x110 & 0x2 & 0x1 & m &     & 0xc &     &     &     &     &     &     \\
0x710 & 0xe & 0x1 & m &     & 0x2 &     &     &     &     &     &     \\
\Xhline{2\arrayrulewidth}
\multicolumn{12}{l}{\textbf{Number of Misses = 16}} \\
\hline
\multicolumn{12}{l}{\textbf{Miss Rate = 0.8}} \\
\Xhline{2\arrayrulewidth}
\end{tabular}
\caption{Direct-Mapped Cache Contents Over Time}
\end{figure}

\begin{figure}[H]
\centering
{\setlength{\tabcolsep}{2pt}
\begin{tabular}{@{\extracolsep{3pt}}cccccccccccc@{}}
\Xhline{2\arrayrulewidth}
\textbf{Transaction} & & & & \multicolumn{2}{c}{\textbf{Set 0}} & \multicolumn{2}{c}{\textbf{Set 1}} & \multicolumn{2}{c}{\textbf{Set 2}} & \multicolumn{2}{c}{\textbf{Set 3}} \\
\cline{5-6}
\cline{7-8}
\cline{9-10}
\cline{11-12}
\textbf{Address} & \textbf{tag} & \textbf{idx} & \textbf{m/h} & \textbf{Way 0} & \textbf{Way 1} & \textbf{Way 0} & \textbf{Way 1} & \textbf{Way 0} & \textbf{Way 1} & \textbf{Way 0} & \textbf{Way 1} \\
\Xhline{2\arrayrulewidth}
0x024 & 0x0 & 0x2 & m &  -  &  -  &  -  &  -  &  -  &  -  &  -  &  -  \\
0x030 & 0x0 & 0x3 & m &     &     &     &     & 0x0 &     &     &     \\
0x07c & 0x1 & 0x3 & m &     &     &     &     &     &     & 0x0 &     \\
0x070 & 0x1 & 0x3 & h &     &     &     &     &     &     &     & 0x1 \\
0x100 & 0x4 & 0x0 & m &     &     &     &     &     &     &     &     \\
0x110 & 0x4 & 0x1 & m & 0x4 &     &     &     &     &     &     &     \\
0x204 & 0x8 & 0x0 & m &     &     & 0x4 &     &     &     &     &     \\
0x214 & 0x8 & 0x1 & m &     & 0x8 &     &     &     &     &     &     \\
0x308 & 0xc & 0x0 & m &     &     &     & 0x8 &     &     &     &     \\
0x110 & 0x4 & 0x1 & h & 0xc &     &     &     &     &     &     &     \\
0x114 & 0x4 & 0x1 & h &     &     &     &     &     &     &     &     \\
0x118 & 0x4 & 0x1 & h &     &     &     &     &     &     &     &     \\
0x11c & 0x4 & 0x1 & h &     &     &     &     &     &     &     &     \\
0x410 & 0x10& 0x1 & m &     &     &     &     &     &     &     &     \\
0x110 & 0x4 & 0x1 & h &     &     &     & 0x10&     &     &     &     \\
0x510 & 0x14& 0x1 & m &     &     &     &     &     &     &     &     \\
0x110 & 0x4 & 0x1 & h &     &     &     & 0x14&     &     &     &     \\
0x610 & 0x18& 0x1 & m &     &     &     &     &     &     &     &     \\
0x110 & 0x4 & 0x1 & h &     &     &     & 0x18&     &     &     &     \\
0x710 & 0x1c& 0x1 & m &     &     &     &     &     &     &     &     \\
\Xhline{2\arrayrulewidth}
\multicolumn{12}{l}{\textbf{Number of Misses = 12}} \\
\hline
\multicolumn{12}{l}{\textbf{Miss Rate = 0.6}} \\
\Xhline{2\arrayrulewidth}
\end{tabular}
}
\caption{Two-Way Set-Associative Cache Contents Over Time with LRU Replacement}
\end{figure}

\begin{figure}[H]
\centering
{\setlength{\tabcolsep}{2pt}
\begin{tabular}{@{\extracolsep{3pt}}cccccccccccc@{}}
\Xhline{2\arrayrulewidth}
\textbf{Transaction} & & & & \multicolumn{2}{c}{\textbf{Set 0}} & \multicolumn{2}{c}{\textbf{Set 1}} & \multicolumn{2}{c}{\textbf{Set 2}} & \multicolumn{2}{c}{\textbf{Set 3}} \\
\cline{5-6}
\cline{7-8}
\cline{9-10}
\cline{11-12}
\textbf{Address} & \textbf{tag} & \textbf{idx} & \textbf{m/h} & \textbf{Way 0} & \textbf{Way 1} & \textbf{Way 0} & \textbf{Way 1} & \textbf{Way 0} & \textbf{Way 1} & \textbf{Way 0} & \textbf{Way 1} \\
\Xhline{2\arrayrulewidth}
0x024 & 0x0 & 0x2 & m &  -  &  -  &  -  &  -  &  -  &  -  &  -  &  -  \\
0x030 & 0x0 & 0x3 & m &     &     &     &     & 0x0 &     &     &     \\
0x07c & 0x1 & 0x3 & m &     &     &     &     &     &     & 0x0 &     \\
0x070 & 0x1 & 0x3 & h &     &     &     &     &     &     &     & 0x1 \\
0x100 & 0x4 & 0x0 & m &     &     &     &     &     &     &     &     \\
0x110 & 0x4 & 0x1 & m & 0x4 &     &     &     &     &     &     &     \\
0x204 & 0x8 & 0x0 & m &     &     & 0x4 &     &     &     &     &     \\
0x214 & 0x8 & 0x1 & m &     & 0x8 &     &     &     &     &     &     \\
0x308 & 0xc & 0x0 & m &     &     &     & 0x8 &     &     &     &     \\
0x110 & 0x4 & 0x1 & h & 0xc &     &     &     &     &     &     &     \\
0x114 & 0x4 & 0x1 & h &     &     &     &     &     &     &     &     \\
0x118 & 0x4 & 0x1 & h &     &     &     &     &     &     &     &     \\
0x11c & 0x4 & 0x1 & h &     &     &     &     &     &     &     &     \\
0x410 & 0x10& 0x1 & m &     &     &     &     &     &     &     &     \\
0x110 & 0x4 & 0x1 & h &     &     &     & 0x10&     &     &     &     \\
0x510 & 0x14& 0x1 & m &     &     &     &     &     &     &     &     \\
0x110 & 0x4 & 0x1 & m &     &     & 0x14&     &     &     &     &     \\
0x610 & 0x18& 0x1 & m &     &     &     &     &     &     &     &     \\
0x110 & 0x4 & 0x1 & m &     &     &     & 0x18&     &     &     &     \\
0x710 & 0x1c& 0x1 & m &     &     &     &     &     &     &     &     \\
\Xhline{2\arrayrulewidth}
\multicolumn{12}{l}{\textbf{Number of Misses = 14}} \\
\hline
\multicolumn{12}{l}{\textbf{Miss Rate = 0.7}} \\
\Xhline{2\arrayrulewidth}
\end{tabular}
}
\caption{Two-Way Set-Associative Cache Contents Over Time with FIFO Replacement}
\end{figure}

\subsection{Sequential Tag Check then Memory Access}
\begin{figure}[H]
\centering
\begin{tabular}{@{\extracolsep{3pt}}llr@{}}
\hline
\textbf{Component} & \textbf{Delay Equation} & \textbf{Delay($\tau$)} \\
\hline
addr\_reg\_M0 & 1 & 1 \\
tag\_decoder & 3 + 2$\times$2 & 7 \\
tag\_mem 	& 10 + [(4+27)/16] & 12 \\
tag\_cmp		& 3 + 2[log2(26)] & 13 \\
tag\_and		& 2 - 1 & 1\\
data\_decoder& 3 + 2$\times$3 & 9 \\
data\_mem	& 10 + [(8+128)/16] & 19 \\
rdata\_mux	& 3[log2(4)] + [32/8] & 10 \\
rdata\_reg\_M1& 1 & 1 \\
\hline
\textbf{Total} & & \textbf{73} \\
\hline
addr\_reg\_M0 & 1 & 1 \\
tag\_decoder & 3 + 2$\times$2 & 7 \\
tag\_mem 	& 10 + [(4+27)/16] & 12 \\
tag\_cmp		& 3 + 2[log2(26)] & 13 \\
tag\_and		& 2 - 1 & 1\\
data\_decoder& 3 + 2$\times$3 & 9 \\
data\_mem	& 10 + [(8+128)/16] & 19 \\
\hline
\textbf{Total} & & \textbf{62} \\
\hline
\end{tabular}
\captionsetup{justification=centering}
\caption{Critical Path and Cycle Time for 2-Way Set-Associative Cache with 
\\ Serialized Tag Check before Data Access}
\end{figure}
The reason that the 2-way set-associative microarchitecture is slower than the direct-mapped microarchitecture is the need for the tag check result to go through the data\_decoder. It happens that the data\_decoder?s delay is relatively significant (9$\tau$). This connection is needed so that the data can be outputted from the correct way. 

\subsection{Parallel Read Hit Path}
\begin{figure}[H]
\centering
\begin{tabular}{@{\extracolsep{3pt}}llr@{}}
\hline
\textbf{Component} & \textbf{Delay Equation} & \textbf{Delay($\tau$)} \\
\hline
addr\_reg\_M0 & 1 & 1 \\
addr\_mux	& 3[log2(2)] + [5/8] & 4 \\
data\_decoder& 3 + 2$\times$3 & 9 \\
data\_mem	& 10 + [(8+128)/16] & 19 \\
rdata\_mux	& 3[log2(4)] + [32/8] & 10 \\
rdata\_reg\_M1& 1 & 1 \\
\hline
\textbf{Total} & & \textbf{44} \\
\hline
\end{tabular}
\caption{Critical Path and Cycle Time for Direct Mapped Cache with
\\ Parallel Read Hit}
\end{figure}
\begin{figure}[H]
\centering
\begin{tabular}{@{\extracolsep{3pt}}llr@{}}
\hline
\textbf{Component} & \textbf{Delay Equation} & \textbf{Delay($\tau$)} \\
\hline
addr\_reg\_M0 & 1 & 1 \\
addr\_mux	& 3[log2(2)] + [5/8] & 4 \\
data\_decoder& 3 + 2$\times$2 & 7 \\
data\_mem	& 10 + [(8+128)/16] & 19 \\
rdata\_mux	& 3[log2(4)] + [32/8] & 10 \\
way\_mux		& 3[log2(2)] + [32/8] & 7 \\
rdata\_reg\_M1& 1 & 1 \\
\hline
\textbf{Total} & & \textbf{49} \\
\hline
\end{tabular}
\caption{Critical Path and Cycle Time for 2-Way Set-Associative Cache with
\\ Parallel Read Hit}
\end{figure}
The reason that the 2-way set-associative microarchitecture is slower than the direct-mapped microarchitecture is the way\_mux, which is needed to output the data from the correct way. This mux has a delay of 7$\tau$, which is relatively significant. 

\subsection{Pipelined Write Hit Path}
\begin{figure}[H]
\centering
\begin{tabular}{@{\extracolsep{3pt}}llr@{}}
\hline
\textbf{Component} & \textbf{Delay Equation} & \textbf{Delay($\tau$)} \\
\hline
addr\_reg\_M0 & 1 & 1 \\
tag\_decoder & 3 + 2$\times$3 & 9 \\
tag\_mem 	& 10 + [(8+26)/16] & 13 \\
tag\_cmp		& 3 + 2[log2(25)] & 13 \\
tag\_and		& 2 - 1 & 1\\
wen\_and		& 2 - 1 & 1\\
wen\_reg\_M1	& 1 & 1\\
\hline
\textbf{Total} & & \textbf{39} \\
\hline
\end{tabular}
\caption{Critical Path and Cycle Time for Direct Mapped Cache with
\\ Pipelined Write Hit}
\end{figure}
\begin{figure}[H]
\centering
\begin{tabular}{@{\extracolsep{3pt}}llr@{}}
\hline
\textbf{Component} & \textbf{Delay Equation} & \textbf{Delay($\tau$)} \\
\hline
addr\_reg\_M0 & 1 & 1 \\
tag\_decoder & 3 + 2$\times$2 & 7 \\
tag\_mem 	& 10 + [(4+27)/16] & 12 \\
tag\_cmp		& 3 + 2[log2(25)] & 13 \\
tag\_and		& 2 - 1 & 1\\
wen\_and		& 2 - 1 & 1\\
wen\_reg\_M1	& 1 & 1\\
\hline
\textbf{Total} & & \textbf{36} \\
\hline
\end{tabular}
\caption{Critical Path and Cycle Time for 2-Way Set-Associative Cache with
\\ Pipelined Write Hit}
\end{figure}

\subsection{Average Memory Access Latency}
\begin{figure}[H]
\centering
\begin{tabular}{@{\extracolsep{3pt}}ccccccc@{}}
\Xhline{2\arrayrulewidth}
& & & \textbf{Hit} & \textbf{Miss} & \textbf{Miss} & \\
& & \textbf{Replacement} & \textbf{Time} & \textbf{Rate} & \textbf{Penalty} & \textbf{AMAL} \\
\textbf{Associativity} & \textbf{$\mu$arch} & \textbf{Policy} & \textbf{($\tau$)} & \textbf{(ratio)} & \textbf{($\tau$)} & \textbf{($\tau$)} \\
\Xhline{2\arrayrulewidth}
Direct Mapped   & Seq & n/a & 68 & 0.8 & 300 & 308\\
2-way Set Assoc & Seq & LRU & 73 & 0.6 & 300 & 253\\
2-way Set Assoc & Seq & FIFO& 73 & 0.7 & 300 & 283\\
Direct Mapped   & PP  & n/a & 44 & 0.8 & 300 & 284\\
2-way Set Assoc & PP  & LRU & 49 & 0.6 & 300 & 229\\
2-way Set Assoc & PP  & FIFO& 49 & 0.7 & 300 & 259\\
\Xhline{2\arrayrulewidth}
\end{tabular}
\caption{Average Memory Access Latency for Six Cache Configurations}
\end{figure}

\cleardoublepage
\section{Array vs. List Cache Behavior}
\begin{figure}[H]
\centering
\begin{tabular}{@{\extracolsep{3pt}}lccccccc@{}}
\Xhline{2\arrayrulewidth}
& & & & \multicolumn{4}{c}{\textbf{CPI Breakdown}}\\
\cline{5-8}
& \textbf{Number of} & & \textbf{Execution} & \textbf{Useful} & \textbf{Raw} & \textbf{Control} & \textbf{Memory} \\
\textbf{Part} & \textbf{Instructions} & \textbf{CPI} & \textbf{Time (cyc)} & \textbf{Work} & \textbf{Stalls} & \textbf{Squashes} & \textbf{Stalls}\\
\hline
\textbf{Part 4.A} & 512 & 1.5 & 768 & 1 & 0.25 & 0 & 0.25\\
\hline
\textbf{Part 4.B} & & & & & & & \\
\Xhline{2\arrayrulewidth}
\end{tabular}
\caption{Execution Time for Reverse Operation on Array and Linked List Data Structures}
\end{figure}

For the cycle type:\\
u = cycle of useful work\\
r = cycle lost due to RAW stal\\
m = cycle lost due to memory stall\\
c = cycle lost due to control squashes\\

\subsection{Analyzing Performance of an Array Data Structure}
Table on next page.\\
The loop runs 64 times. Therefore, the total number of instructions executed is 64 $\times$ 8 = 512 instructions. 
The first iteration (between the first two bold lines) shows the pipeline flow when the cache misses both loads. Because the cachline is 16 bytes, these misses will occur every 4 loops. The number of cycles in this cache-miss loop is 18. For when both load words find a hit in the cache (between the latter two bold lines), the cycle count for the loop is 10. This occurs 3 times out of every 4 loops. Therefore, the total cycles for 4 loops is 18 + 3$\times$10 = 48. This is then looped 16 times for a total of 64 loops. Therefore, the total number of cycles for the program (ignoring the first 4 instructions for setup) is 16 $\times$ 48 = 768. The CPI for the entire program is therefore 768/512 = 1.5.
Here is a breakdown of the CPI for each cycle type:\\
\begin{lstlisting}
First Iteration Cycle Type Breakdown:
u = 8 cycles
m = 8 cycles
r = 0 cycle
c = 2 cycles
Second Iteration Cycle Type Breakdown:
u = 8 cycles
m = 0 cycles
r = 0 cycle
c = 2 cycles
Overall CPI Breakdown:
u = 16*8 + 48*8 = 512 cycles, 512/512 = 1.00 CPI
m = 16*8 + 48*0 = 128 cycles, 128/512 = 0.25 CPI
r = 16*0 + 48*0 =   0 cycles,   0/512 = 0.00 CPI
c = 16*2 + 48*2 = 128 cycles, 128/512 = 0.25 CPI
total                                 = 1.50 CPI
\end{lstlisting}
\begin{landscape}
\begin{figure}[H]
\centering
{\setlength{\tabcolsep}{2pt}
\begin{tabular}{|l|c|c|c|c!{\vrule width 1.5pt}c|c|c|c|c|c|c|c|c|c|c|c|c|c|c|c|c|c!{\vrule width 2pt}c|c|c|c|c|c|c|c|c|c!{\vrule width 2pt}c|}
\hline
Cycle type: &  &  &  &  & m & m & m & m & u & m & m & m & m & u & u & u & u & u & u & u & c & c & u & u & u & u & u & u & u & u & c & c & u \\ \hline
Instruction & 1 & 2 & 3 & 4 & 5 & 6 & 7 & 8 & 9 & 10 & 11 & 12 & 13 & 14 & 15 & 16 & 17 & 18 & 19 & 20 & 21 & 22 & 23 & 24 & 25 & 26 & 27 & 28 & 29 & 30 & 31 & 32 & 33 \\ \hline
lw r12, 0(r4) & F & D & X & M & M & M & M & M & W &  &  &  &  &  &  &  &  &  &  &  &  &  &  &  &  &  &  &  &  &  &  &  &  \\ \hline
lw r13, 0(r5) &  & F & D & X & X & X & X & X & M & M & M & M & M & W &  &  &  &  &  &  &  &  &  &  &  &  &  &  &  &  &  &  &  \\ \hline
sw r12, 0(r5) &  &  & F & D & D & D & D & D & X & X & X & X & X & M & W &  &  &  &  &  &  &  &  &  &  &  &  &  &  &  &  &  &  \\ \hline
sw r13, 0(r4) &  &  &  & F & F & F & F & F & D & D & D & D & D & X & M & W &  &  &  &  &  &  &  &  &  &  &  &  &  &  &  &  &  \\ \hline
addiu r14, r5, 0 &  &  &  &  &  &  &  &  & F & F & F & F & F & D & X & M & W &  &  &  &  &  &  &  &  &  &  &  &  &  &  &  &  \\ \hline
addiu r4, r4, 4 &  &  &  &  &  &  &  &  &  &  &  &  &  & F & D & X & M & W &  &  &  &  &  &  &  &  &  &  &  &  &  &  &  \\ \hline
addiu r5, r5, -4 &  &  &  &  &  &  &  &  &  &  &  &  &  &  & F & D & X & M & W &  &  &  &  &  &  &  &  &  &  &  &  &  &  \\ \hline
bne r4, r14, loop &  &  &  &  &  &  &  &  &  &  &  &  &  &  &  & F & D & X & M & W &  &  &  &  &  &  &  &  &  &  &  &  &  \\ \hline
opA &  &  &  &  &  &  &  &  &  &  &  &  &  &  &  &  & F & D & - & - & - &  &  &  &  &  &  &  &  &  &  &  &  \\ \hline
opB &  &  &  &  &  &  &  &  &  &  &  &  &  &  &  &  &  & F & - & - & - & - &  &  &  &  &  &  &  &  &  &  &  \\ \hline
lw r12, 0(r4) &  &  &  &  &  &  &  &  &  &  &  &  &  &  &  &  &  &  & F & D & X & M & W &  &  &  &  &  &  &  &  &  &  \\ \hline
lw r13, 0(r5) &  &  &  &  &  &  &  &  &  &  &  &  &  &  &  &  &  &  &  & F & D & X & M & W &  &  &  &  &  &  &  &  &  \\ \hline
sw r12, 0(r5) &  &  &  &  &  &  &  &  &  &  &  &  &  &  &  &  &  &  &  &  & F & D & X & M & W &  &  &  &  &  &  &  &  \\ \hline
sw r13, 0(r4) &  &  &  &  &  &  &  &  &  &  &  &  &  &  &  &  &  &  &  &  &  & F & D & X & M & W &  &  &  &  &  &  &  \\ \hline
addiu r14, r5, 0 &  &  &  &  &  &  &  &  &  &  &  &  &  &  &  &  &  &  &  &  &  &  & F & D & X & M & W &  &  &  &  &  &  \\ \hline
addiu r4, r4, 4 &  &  &  &  &  &  &  &  &  &  &  &  &  &  &  &  &  &  &  &  &  &  &  & F & D & X & M & W &  &  &  &  &  \\ \hline
addiu r5, r5, -4 &  &  &  &  &  &  &  &  &  &  &  &  &  &  &  &  &  &  &  &  &  &  &  &  & F & D & X & M & W &  &  &  &  \\ \hline
bne r4, r14, loop &  &  &  &  &  &  &  &  &  &  &  &  &  &  &  &  &  &  &  &  &  &  &  &  &  & F & D & X & M & W &  &  &  \\ \hline
opA &  &  &  &  &  &  &  &  &  &  &  &  &  &  &  &  &  &  &  &  &  &  &  &  &  &  & F & D & - & - & - &  &  \\ \hline
opB &  &  &  &  &  &  &  &  &  &  &  &  &  &  &  &  &  &  &  &  &  &  &  &  &  &  &  & F & - & - & - & - &  \\ \hline
lw r12, 0(r4) &  &  &  &  &  &  &  &  &  &  &  &  &  &  &  &  &  &  &  &  &  &  &  &  &  &  &  &  & F & D & X & M & W \\ \hline
\end{tabular}
}
\caption{Array Data Structure Pipeline Diagram}
\end{figure}
\end{landscape}

\subsection{Analyzing Performance of an Array Data Structure}
Table on next page.\\
The loop runs 64 times. Therefore, the total number of instructions executed is 64 $\times$ 8 = 512 instructions. 
The first iteration (between the first two bold lines) shows the pipeline flow when the cache misses both loads. Because the cachline is 16 bytes, these misses will occur every 4 loops. The number of cycles in this cache-miss loop is 18. For when both load words find a hit in the cache (between the latter two bold lines), the cycle count for the loop is 10. This occurs 3 times out of every 4 loops. Therefore, the total cycles for 4 loops is 18 + 3$\times$10 = 48. This is then looped 16 times for a total of 64 loops. Therefore, the total number of cycles for the program (ignoring the first 4 instructions for setup) is 16 $\times$ 48 = 768. The CPI for the entire program is therefore 768/512 = 1.5.
Here is a breakdown of the CPI for each cycle type:\\
\begin{lstlisting}
First Iteration Cycle Type Breakdown:
u = 8 cycles
m = 8 cycles
r = 0 cycle
c = 2 cycles
Second Iteration Cycle Type Breakdown:
u = 8 cycles
m = 0 cycles
r = 0 cycle
c = 2 cycles
Overall CPI Breakdown:
u = 16*8 + 48*8 = 512 cycles, 512/512 = 1.00 CPI
m = 16*8 + 48*0 = 128 cycles, 128/512 = 0.25 CPI
r = 16*0 + 48*0 =   0 cycles,   0/512 = 0.00 CPI
c = 16*2 + 48*2 = 128 cycles, 128/512 = 0.25 CPI
total                                 = 1.50 CPI
\end{lstlisting}
\begin{landscape}
\begin{figure}[H]
\centering
{\setlength{\tabcolsep}{2pt}
\begin{tabular}{|l|c|c|c|c!{\vrule width 1.5pt}c|c|c|c|c|c|c|c|c|c|c|c|c|c|c|c|c|c!{\vrule width 2pt}c|c|c|c|c|c|c|c|c|c!{\vrule width 2pt}c|}
\hline
Cycle type: &  &  &  &  & m & m & m & m & u & m & m & m & m & u & u & u & u & u & u & u & c & c & u & u & u & u & u & u & u & u & c & c & u \\ \hline
Instruction & 1 & 2 & 3 & 4 & 5 & 6 & 7 & 8 & 9 & 10 & 11 & 12 & 13 & 14 & 15 & 16 & 17 & 18 & 19 & 20 & 21 & 22 & 23 & 24 & 25 & 26 & 27 & 28 & 29 & 30 & 31 & 32 & 33 \\ \hline
lw r12, 0(r4) & F & D & X & M & M & M & M & M & W &  &  &  &  &  &  &  &  &  &  &  &  &  &  &  &  &  &  &  &  &  &  &  &  \\ \hline
lw r13, 0(r5) &  & F & D & X & X & X & X & X & M & M & M & M & M & W &  &  &  &  &  &  &  &  &  &  &  &  &  &  &  &  &  &  &  \\ \hline
sw r12, 0(r5) &  &  & F & D & D & D & D & D & X & X & X & X & X & M & W &  &  &  &  &  &  &  &  &  &  &  &  &  &  &  &  &  &  \\ \hline
sw r13, 0(r4) &  &  &  & F & F & F & F & F & D & D & D & D & D & X & M & W &  &  &  &  &  &  &  &  &  &  &  &  &  &  &  &  &  \\ \hline
addiu r14, r5, 0 &  &  &  &  &  &  &  &  & F & F & F & F & F & D & X & M & W &  &  &  &  &  &  &  &  &  &  &  &  &  &  &  &  \\ \hline
addiu r4, r4, 4 &  &  &  &  &  &  &  &  &  &  &  &  &  & F & D & X & M & W &  &  &  &  &  &  &  &  &  &  &  &  &  &  &  \\ \hline
addiu r5, r5, -4 &  &  &  &  &  &  &  &  &  &  &  &  &  &  & F & D & X & M & W &  &  &  &  &  &  &  &  &  &  &  &  &  &  \\ \hline
bne r4, r14, loop &  &  &  &  &  &  &  &  &  &  &  &  &  &  &  & F & D & X & M & W &  &  &  &  &  &  &  &  &  &  &  &  &  \\ \hline
opA &  &  &  &  &  &  &  &  &  &  &  &  &  &  &  &  & F & D & - & - & - &  &  &  &  &  &  &  &  &  &  &  &  \\ \hline
opB &  &  &  &  &  &  &  &  &  &  &  &  &  &  &  &  &  & F & - & - & - & - &  &  &  &  &  &  &  &  &  &  &  \\ \hline
lw r12, 0(r4) &  &  &  &  &  &  &  &  &  &  &  &  &  &  &  &  &  &  & F & D & X & M & W &  &  &  &  &  &  &  &  &  &  \\ \hline
lw r13, 0(r5) &  &  &  &  &  &  &  &  &  &  &  &  &  &  &  &  &  &  &  & F & D & X & M & W &  &  &  &  &  &  &  &  &  \\ \hline
sw r12, 0(r5) &  &  &  &  &  &  &  &  &  &  &  &  &  &  &  &  &  &  &  &  & F & D & X & M & W &  &  &  &  &  &  &  &  \\ \hline
sw r13, 0(r4) &  &  &  &  &  &  &  &  &  &  &  &  &  &  &  &  &  &  &  &  &  & F & D & X & M & W &  &  &  &  &  &  &  \\ \hline
addiu r14, r5, 0 &  &  &  &  &  &  &  &  &  &  &  &  &  &  &  &  &  &  &  &  &  &  & F & D & X & M & W &  &  &  &  &  &  \\ \hline
addiu r4, r4, 4 &  &  &  &  &  &  &  &  &  &  &  &  &  &  &  &  &  &  &  &  &  &  &  & F & D & X & M & W &  &  &  &  &  \\ \hline
addiu r5, r5, -4 &  &  &  &  &  &  &  &  &  &  &  &  &  &  &  &  &  &  &  &  &  &  &  &  & F & D & X & M & W &  &  &  &  \\ \hline
bne r4, r14, loop &  &  &  &  &  &  &  &  &  &  &  &  &  &  &  &  &  &  &  &  &  &  &  &  &  & F & D & X & M & W &  &  &  \\ \hline
opA &  &  &  &  &  &  &  &  &  &  &  &  &  &  &  &  &  &  &  &  &  &  &  &  &  &  & F & D & - & - & - &  &  \\ \hline
opB &  &  &  &  &  &  &  &  &  &  &  &  &  &  &  &  &  &  &  &  &  &  &  &  &  &  &  & F & - & - & - & - &  \\ \hline
lw r12, 0(r4) &  &  &  &  &  &  &  &  &  &  &  &  &  &  &  &  &  &  &  &  &  &  &  &  &  &  &  &  & F & D & X & M & W \\ \hline
\end{tabular}
}
\caption{Array Data Structure Pipeline Diagram}
\end{figure}
\end{landscape}

\end{document}